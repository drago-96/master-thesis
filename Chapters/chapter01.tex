\chapter{Mathematical background}

The theory of elliptic curves is at the foundation of isogeny-based Cryptography, but in a very different style from classical Elliptic Curve Cryptography: the latter uses the group of points of a single curve, while the first uses graphs of elliptic curves.

Our introduction on these topics will follow \cite{Silverman} and \cite{DeFeo_intro}.

\section{Elliptic curves}

We start by giving the most general definition of an elliptic curve over a field $K$.

\begin{definition}
    An \emph{elliptic curve} $E$ \emph{over} $K$ is a smooth projective curve of genus $1$, with a given $K$-rational point.
\end{definition}

Since we will work over finite fields of large characteristic, we will assume in all our theorems that $\car K\neq 2,3$. This allows us to simplify the representation of an elliptic curve.

\begin{proposition}[Weierstrass model]
    Let $E$ be an elliptic curve over $K$, with $\car K\neq 2,3$. Then $E$ is isomorphic to the curve $$ZY^2=X^3+aXZ^2+bZ^2$$ with $a,b\in K$ and $4a^3+27b^2\neq0$.
    
    The point $\Oc=(0:1:0)$ is called the \emph{point at infinity} of $E$, and it's the only point on $Z=0$.
\end{proposition}

Usually we then identify the curve with its affine patch at $Z\neq0$, i.e. with the equation $$y^2=x^3+ax+b,$$
plus the point at infinity $\Oc$.

Is is well known that any elliptic curve has a structure of abelian group, with $\Oc$ as the neutral element.

\section{Isogenies}

\begin{proposition}
    Group map iff algebraic map that sends infinity to infinity.
\end{proposition}

Automorphisms.

Separable and inseparable. Splitting lemma.

Multiplication by $m$ and structure of $E[m]$.

Dual isogeny.

\section{Elliptic curves over finite fields}

Frobenius.

Hasse bounds.

(Weil conjectures)

Twists

\section{Supersingularity}

Equivalent conditions.

(Bounds V.4.7)

Special supersingular j-invariants.

How to compute ss ec.

\section{Quaternion algebras}

\section{Deuring correspondence}


