\chapter{Universal Composability}

\section{Introduction}

\section{Instantiation of the framework}

\section{The composition theorem}
Only statement, with pictures.

\section{The Simpler UC}
We will use the ``Simpler Universal Composability" framework, also defined by Canetti.

In this framework the number of parties is fixed, and they communicate with each other (and possibly with an ideal functionality) through a router controlled by the adversary.

The communication is supposed to be authenticated, so each message has a \texttt{public header} and a \texttt{private content}.

Formally, the adversary can only:
\begin{itemize}
    \item Read sender and recipient of each message
    \item Read the \texttt{public header} of each message
    \item Decide when and if deliver each message to the recipient
\end{itemize}

[picture with router]

Moreover the adversary can corrupt any parties they want, formally by sending a $(\texttt{corrupt}, P_i)$ message through the router. The corrupted party will then follow the adversary instructions, and the environment will be notified.

We will mostly be dealing with \emph{static} corruptions, which means that the adversary can only corrupt a fixed set of parties at the start of the protocol.

\begin{definition}
    Let $\pi$ be a protocol and $\mathcal F$ be an ideal functionality. We say that $\pi$ \emph{SUC-securely computes} $\mathcal F$ if for any PPT adversary $\adv$ there exists a PPT simulator $\sdv$ such that for any PPT environment $\zdv$ it holds that
    $$ \{ \text{SUC-IDEAL}_{\Fun, \sdv, \zdv} \} \cindist \{ \text{SUC-REAL}_{\pi, \adv, \zdv} \}$$
\end{definition}

We observe that for any adversary $\adv$ and environment $\zdv$ there exists an environment $\zdv'$ that includes internally the adversary $\adv$ and communicates with a dummy adversary that simply relays messages between $\zdv'$ and the router.

Since the definition quantifies over every adversary and environment, we can construct a simulator only for the dummy adversary.
