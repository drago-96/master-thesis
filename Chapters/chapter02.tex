\chapter{Isogeny Based Cryptography}

In this chapter we will explain how isogenies of supersingular elliptic curves can be used for cryptography, and in particular for post-quantum secure protocols.

We will introduce the two main frameworks of isogeny crypto, namely SIDH and CSIDH, analyzing their inner workings, their hardness assumptions and their post-quantum security.

Some of the most useful resources on the subject are \cite{DeFeo_intro}, \url{https://defeo.lu/hdr} and \cite{Galbraith_SIproblems}.

\section{Diffie-Hellman key exchange}

One of the most important primitives in cryptography is the one of \emph{key exchange}, which allows two parties to obtain a shared secret between them while only communicating on public channels.

The main prototype of key exchanges is the one proposed by Diffie and Hellman in their seminal paper \cite{DH}, which marks the birth of the whole field of public key cryptography.

The DH protocol is very minimal, also in its setup: we fix a prime number $p$ and a generator $g$ of $\F_p^\ast$. Then the two parties Alice and Bob choose random values $a$ and $b$ respectively, and send each other the public values $A=g^a$ and $B=g^b$. The shared secret is then $g^{ab}$, which can be computed by Alice and Bob as $B^a$ and $A^b$, respectively.

This protocol is secure since solving the \emph{discrete logarithm problem} is hard, in the sense that there is no algorithm that solves it in $O(\text{poly}(\log p))$ operations; for example, the generic baby-step giant-step algorithm has a running time of $O(\sqrt p)$, while the most optimized General Number Field Sieve has complexity $L_p[1/3,\sqrt[3]{64/9}]$\footnote{Recall that $L_n[\alpha, c]=\exp\left( (c+o(1))\log^\alpha(n)\log\log^{1-\alpha}(n) \right)$}.

Actually, since P vs NP is still an open problem, the whole field of cryptography works with assumptions and reductions: there is a small set of problems that are considered ``hard", and the proofs of security are simply reductions to one of those problems; more explicitly, most proofs use a format similar to ``suppose an adversary $\adv$ can break protocol $\pi$; then with the following steps $\adv$ can be used to solve an instance of the hard problem X".

We will not go into more details here (for example what \emph{break} means), since in the next chapter we will introduce the Universal Composability framework, which is one of the most general definitions of security for generic cryptographic protocols.

However we notice that for Diffie-Hellman there are at least two assumptions that can be made:
\begin{assumption}[DDH]
    The distributions $(g,g^a,g^b,g^{ab})$ and $(g,g^a,g^b,g^c)$ are computationally indistinguishable for random $a,b,c$, i.e. a polynomial-time adversary has a negligible probability of distinguishing them.s
\end{assumption}
\begin{assumption}[CDH]
    Any polynomial-time adversary has a negligible probability of computing $g^{ab}$ from the knowledge of $g,g^a,g^b$.
\end{assumption}

For the naive version of the protocol, we notice that the decisional DH assumption can easily be broken with the use of Legendre symbols: if $g^a,g^b$ are both non residues, then $g^{ab}$ also isn't a square, while $g^c$ takes both square and non-square values.

Indeed one of the most useful abstractions of DH uses a group $G$ of prime order $q$, with a generator $g$. For finite field Diffie-Hellman this means taking $p=2q+1$ a Sophie-Germain prime, and $G={\F_p^\ast}^2$.

Another instantiation of Diffie-Hellman uses elliptic curves: we choose a curve $E$ over some finite field $\F_p$, we take a point $G\in E(\F_p)$ with prime order and use $G=\langle P\rangle$, the subgroup generated by $P$. In this case the operation $g^a$ becomes the scalar multiplication $[a]P$.

Those two protocols, called DH and ECDH, have been the core of public key cryptography, together with RSA, right from their discovery up until today. Indeed, any browser that visits a HTTPS website will probably check an RSA signature of the certificate, and then perform an ephemeral ECDH key exchange before encrypting all traffic with the newly obtained shared key.

However all instantiations of the DH protocol are now ``broken": in 1994 Shor published a polynomial-time quantum algorithm \cite{Shor94} that solves the discrete logarithm problem in any group. That day a race began between cryptographers trying to find \emph{post-quantum} alternatives, and many actors trying to actually build quantum computers able to perform discrete log computations on cryptographically sized input.

In the next sections we will thus introduce some new key exchange protocols which are based on isogenies of supersingular elliptic curves, whose problems are conjectured to be hard even for quantum computers.

In particular we will notice how CSIDH uses the action of an abelian group instead of group operations, while SIDH completely parts ways with groups and commutativity. A great introduction on these same topics can be found in \cite{Smith_DH}.


\section{History of isogeny-based cryptography}
With the discovery of Shor's order-finding algorithm for quantum computers in 1994 all public key cryptography known at the time was suddenly broken; even if not concretely, the presence of a polynomial quantum algorithm for what was believed a ``hard" problem was a shock to many computer scientists and cryptographers.

The roots of isogeny based cryptography lie shortly after Shor's algorithm: Jean-Marc Couveignes gave a talk in 1997 describing the first isogeny-based protocol, but it was published only ten years later in \cite{Couveignes}. It introduces the fundamental concept of \emph{Hard Homogeneous Spaces}, which is a generalization of Diffie-Hellman-like key exchanges using group \emph{actions} instead of the usual group operations.

Independently, Rostovtsev and Stolbunov proposed in \cite{Rostovtsev} a similar protocol using walks on isogeny graphs, or more precisely walks on the Cayley graph of the class group of some endomorphism ring.

Those papers introduced the main ideas of isogeny based cryptography, but were missing one fundamental detail: the use of \emph{supersingular} elliptic curves, both for faster isogeny evaluation and for improved security.\todo{SS curves are bad in ECC?}

Indeed the true birth of isogeny crypto can be placed with the introduction of a hash function by Charles, Goren and Lauter in \cite{CGL}. They used the graph of supersingular elliptic curves over $\F_{p^2}$ with $\ell$-isogenies, proving its ``fast-mixing" properties.

In 2011 De Feo and Jao (and later Plut) introduced the \emph{Supersingular Isogeny Diffie-Hellman} protocol \cite{SIDH}, which is probably the most famous isogeny protocol, since it's the only one that has been submitted to the NIST competition for post-quantum key exchanges algorithms, with the name of SIKE.

Another major breakthrough in isogeny crypto has been the implementation of the Couveignes protocol with supersingular curves, which led to the CSIDH protocol \cite{CSIDH}.

In the recent years there has been an exponential growth of isogeny based constructions for many different cryptographic primitives: mainly signatures (SeaSign, CSI-FiSh, SQISign), but also VDFs and other MPC-friendly primitives like Oblivious Transfer, which will be our focus.\todo{cite all}

\section{Supersingular Isogeny Diffie-Hellman}
In this section we finally describe the SIDH protocol, introduced by Jao and De Feo in \cite{SIDH11}. Besides the actual paper, \cite{Costello_SIKE} is a tutorial for SIKE with all the computations for a toy example.

The high level view of this protocol can be understood with the following diagram:
\[\begin{tikzcd}
E \rar["\phi"] \dar["\psi"] & E/\langle P\rangle \dar\\
E/\langle Q\rangle \rar & E/\langle P, Q\rangle
\end{tikzcd}\]

Where $E$ is a supersingular elliptic curve, the points $P$ and $Q$ are the secrets of Alice and Bob, the two quotients $E/\langle P\rangle$ and $E/\langle Q\rangle$ are the exchanged values and finally $E/\langle P, Q\rangle$ is the shared secret.

Moreover the two isogenies $\phi$ and $\psi$ are computed by a random walk in different degrees isogeny graphs.

The only thing that is not clear from this diagram is how to actually compute the right and bottom isogenies, without publishing $P$ or equivalently $\phi$.

Indeed, all the security of the protocol stands in the fact that it is hard for an attacker to recover the secret isogeny $\phi$, since the following problem is considered hard:
\begin{problem}
    Let $E,E'$ be two isogenous curves. Find an isogeny $\phi: E\to E'$.
\end{problem}

The trick to make the protocol work is to include in the public key some more information about $\phi$ than only its codomain. This is conjectured not to simplify much the isogeny finding problem, but still is a point that requires care, as we will see.

The protocol fixes a prime of the form $p=\ell_A^{e_A}\ell_B^{e_B}\cdot f\pm1$, where $\ell_A,\ell_B$ are small primes (usually $2$ and $3$) and $f$ is a small cofactor. Then it chooses a supersingular elliptic curve $E$ defined over $\F_q=\F_{p^2}$ with order $(\ell_A^{e_A}\ell_B^{e_B} f)^2$.

In this way $E[\ell_A^{e_A}]$ is all defined over $\F_q$, and since it's isomorphic to $\Zn{\ell_A^{e_A}}\times\Zn{\ell_A^{e_A}}$ it has $\ell_A^{e_A-1}(\ell_A+1)$ distinct cyclic subgroups of order $\ell_A^{e_A}$, each of which defines a different $\F_q$-isogeny. Thus the secret in this case is the kernel subgroup.

The other important step is to fix a basis $\{ P_A,Q_A \}$ of $E[\ell_A^{e_A}]$, and a basis $\{ P_B, Q_B \}$ of $E[\ell_B^{e_B}]$.

All these parameters are fixed by the protocol, and the actual key exchange is defined in figure \ref{prot_SIDH}.

\begin{figure}
    \myproc{Protocol SIDH}{
        \textbf{Alice} \> \> \textbf{Bob} \\
        a \sample \Zn{\ell_A^{e_A}} \> \> b \sample \Zn{\ell_B^{e_B}} \\
        A = P_A + [a]Q_A \> \> B = P_B + [b]Q_B \\
        \phi_A: E\to E_A=E/\langle A\rangle \> \> \phi_B: E\to E_B=E/\langle B\rangle\\
        \> \sendmessageright*{E_A, (\phi_A(P_B),\phi_A(Q_B))} \> \\
        \> \sendmessageleft*{E_B, (\phi_B(P_A),\phi_A(Q_A))} \> \\
        B' = \phi_B(P_A) + [a]\phi_B(Q_A) \> \> A' = \phi_A(P_B) + [b]\phi_A(Q_B) \\
        \phi'_A: E_B\to E_{BA}=E_B/\langle A'\rangle \> \> \phi'_B: E_A\to E_{AB}=E_A/\langle B'\rangle\\
        s=j(E_{BA}) \> \> s=j(E_{AB})
    }
    \caption{The SIDH protocol}
    \label{prot_SIDH}
\end{figure}

We can easily see that the protocol is correct, since both compositions $\phi'_A\circ\phi_B:E\to E_{BA}$ and $\phi'_B\circ\phi_A:E\to E_{AB}$ have the same kernel $\langle A,B\rangle$, thus are isomorphic and with the same $j$-invariant.

One of the most important details of this protocol isn't described in our sketched representation, but is the reason why we can actually perform the protocol as the honest users: the isogeny computation.

Indeed, if blindly applying Velu's formula for computing $\phi_A$ results in $O(\ell_A^{e_A})$ operations, which is roughly as much operations as an attacker needs to do in order to break the protocol.

Instead we use the fact that all our isogenies have \emph{smooth} degree, and thus can be computed as the composition of many small degree isogenies. In general, let $R$ be a point of order $\ell^e$ on the curve $E$; we will describe how to compute the isogeny $\phi:E\to E/\langle R\rangle$ and also how to compute the point $\phi(T)$ for any other $T$.

We will do that recursively, and start by setting $E_0=E$, $R_0=R$ and $T_0=T$. Then, for any $0\le i<e$ we compute (this time with Velu's formula) $$E_{i+1}=E_i/\langle [\ell^{e-i-1}R_i] \rangle$$ with the associated isogeny $\phi_i:E_i\to E_{i+1}$, and in particular $R_{i+1}=\phi_i(R_i)$ and $T_{i+1}=\phi_i(T_i)$. In the end, $E/\langle R \rangle=E_e$, and $\phi=\phi_{e-1}\circ\dots\circ\phi_0$.

With this algorithm, we can compute efficiently all the quotients, in time $O(e^2\ell)$. However, we can still do better with the implementation of the so called ``isogeny strategies", which are one of the main improvements of the 2014 version of the paper \cite{SIDH14}.

One final implementation detail is that the curves are in Montgomery form, so that we can efficiently compute additions and isogenies with $x$ only arithmetic.

Indeed in the last 10 years there have been many optimizations to SIKE, from mathematical ones like Montgomery curves to optimizing the assembly instructions for each architecture.

For example, using the implementations on the official submission page \url{https://sike.org/},
we tested that the reference implementation uses around 900 million CPU cycles for the \texttt{KeyGen} algorithm of \texttt{SIKEp434}, while the optimized AMD64 implementation uses only 5 million cycles for the same operation. This means being able to run the three operations \texttt{KeyGen, Encaps, Decaps} in just around 10ms and thus might be a valid option for the TLS key exchange part.

Some more detailed benchmarks of the use of SIKE inside TLS can be found on \href{https://blog.cloudflare.com/the-tls-post-quantum-experiment/}{Cloudflare's} or \href{https://aws.amazon.com/it/blogs/security/round-2-hybrid-post-quantum-tls-benchmarks/}{Amazon's} blog posts.


\subsection{How to break SIDH}
One of the most simple ways to attack SIDH, and in general the isogeny problem, is the \emph{claw finding} algorithm: given the two curves $E$ and $E_A$, build the two trees on the $\ell_A$-isogeny graph starting from $E$ and $E_A$ with depth $e_A/2$; then search if there is a common leaf and join the two half-paths to compute the secret isogeny $\phi_A$.

This attack has a cost of about $O(\ell_A^{e_A/2})$ time and space; since we don't want imbalances in the security for Alice or Bob, the parameters are selected such that $\ell_A^{e_A}\simeq \ell_B^{e_B}\simeq \sqrt{p}$. Indeed, the parameters for the standardized protocol \texttt{SIKEp434} are $p=2^{216}3^{137}-1$.

In terms of quantum security, the claw finding algorithm can be optimized to run in $O(\sqrt[6]{p})$ instead of the classical $O(\sqrt[4]{p})$, as described in \cite{Tani_claw}.

Indeed this algorithm is the best known attack to the ``isogeny with fixed degree" problem, and in general to SIKE. For example, Microsoft has recently published two \href{https://www.microsoft.com/en-us/msrc/sike-cryptographic-challenge}{challenges} with reduced-size instances of SIKE, with bounties of \$5,000 and \$50,000. The smallest of the two has been broken by Udovenko and Vitto \cite{SIKE_challenge} this autumn with some clever optimizations of the claw finding (or meet-in-the-middle) algorithm.


However, SIDH gives to the attacker also the action of the secret isogeny $\phi_A$ on the torsion subgroup $E[\ell_B^{e_B}]$. This is conjectured not to pose a serious threat, but there is an attack by Galbraith et al. \cite{SIKE_Galbraith} in the case that Alice runs many instances of SIDH with the same pair of keys.

We can describe two attack models, in terms of what is the output of Alice; in both cases the attack by Galbraith will recover the secret key $a\in\Zn{\ell_A^{e_A}}$ of Alice:
\begin{itemize}
    \item $O(E,R,S)=E/\langle R+[a]S\rangle$, i.e. Alice will complete her part of the key exchange and then will output the shared secret.
    \item $O(E,R,S,E')$ which outputs whether $j(E')=j(E/\langle R+[a]S\rangle)$; this model is more realistic, if for example Alice uses an authenticated encryption scheme with a key derived from the shared $j$-invariant, and returns an error when the authentication tag mismatches.
\end{itemize}

This attacks means that SIDH can only be used as an interactive and ephemeral key exchange. Indeed the SIKE protocol is the transformation of SID into a KEM protocol by means of the Hofheinz transform \cite{Hofheinz}. Details on the SIKE protocol and this transformation can be found in the official specification at \url{https://sike.org/files/SIDH-spec.pdf}.

\section{Equivalence of SIDH hardness assumptions}
$\ell$-IsogenyPath vs. EndRing.

Deuring correspondence computational aspects and KLPT.

Actual SSCDH and SSDDH.

\section{Commutative SIDH}
Works on $\F_p$, with isogenies defined over $\F_p$.

\subsection{Action of the class group}
$Cl(\Oc)$ acts on $Ell(\Oc)$.

\subsection{The CSIDH protocol}
How to compute action for small norm ideals

\section{CSIDH assumptions}
Not fully post-quantum secure

\cite{breaking_DDH}

