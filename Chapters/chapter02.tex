\chapter{Isogeny Based Cryptography}

Let's now introduce how elliptic curve isogenies are used for cryptography.

There are many hardness assumptions, also depending on the framework used: there are the SIDH-based protocols and the CSIDH-based protocols.

\section{Supersingular Isogeny Diffie-Hellman}
\subsection{Expander graphs}
The $\F_{p^2}$ graph of $\ell$-isogenies

\subsection{The SIDH protocol}
Walk on graph with Velu's formulas.

Use of torsion points for computing the right isogeny.

\section{Equivalence of SID hardness assumptions}
$\ell$-IsogenyPath vs. EndRing.

Deuring correspondence computational aspects and KLPT.

\section{Commutative SIDH}
Works on $\F_p$, with isogenies defined over $\F_p$.

\subsection{Action of the class group}
$Cl(\Oc)$ acts on $Ell(\Oc)$.

\subsection{The CSIDH protocol}
How to compute action for small norm ideals

\section{CSIDH assumptions}
Not fully post-quantum secure

